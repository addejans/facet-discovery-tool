\documentclass{beamer}

\usepackage{amsmath,listings,graphicx,color}

\author[Serge Kruk] { Adam DeJans, Serge Kruk, Jason Medcoff, Blair Sibu}
\title{Automated facet discovery tool}
\institute{Oakland University}


\mode<presentation>
{ \usetheme{Copenhagen} }
\begin{document}
\begin{frame}
  \titlepage
\end{frame}
\lstset{language=Python,label=bin_packing}
\begin{frame}
  \frametitle{Motivation for our work}
  \begin{itemize}
    \item Constraint programming is expressive
    \item Integer programming is fast
  \end{itemize}
\end{frame}

\begin{frame}[fragile]
  \frametitle{Assignment in Constraint programming}
  \begin{lstlisting}
set of int: People = 1..n;
set of int: Tasks = 1..m;
array[People,Tasks] of int: cost;
array[People] of var Tasks: task;
constraint alldifferent(task);
solve minimize sum(w in People) (cost[w,task[w]]);
  \end{lstlisting}
\end{frame}

\begin{frame}[fragile]
  \frametitle{Assignment in Integer Programming}
\begin{lstlisting}{}
set People;
set Tasks;
param c {People,Tasks} > 0;
var X {People, Tasks} binary;
minimize Z: sum {p in People} sum {q in Tasks} 
            c[p,q] * X[p,q];
subject to P1 {p in People}: 
            sum {q in Tasks} X[p,q] = 1;
subject to Q1 {q in Tasks} : 
            sum {p in People} X[p,q] <= 1;
\end{lstlisting}
\end{frame}

\begin{frame}
  \frametitle{Contrast}
  \begin{itemize}
  \item<1-> CP shorter (orders of magnitude)
  \item<2-> CP 'natural' (300+ global constraints)
  \item<3-> IP requires 'tricks' (and tricks vary with solvers)
  \end{itemize}
\end{frame}

\begin{frame}
  \frametitle{Integer programming solves faster}
  \begin{center}
    Sometimes!
  \end{center}
  
  \begin{itemize}
  \item<1-> Google has a library with both CP and IP.
  \item<2-> CP solver time: 475 miliseconds
  \item<3-> IP solver time : 20 miliseconds
  \end{itemize}
\end{frame}

\begin{frame}
  \frametitle{Our objective}
  Automatically translate CP models into efficient IP models.
\end{frame}

\begin{frame}
  \frametitle{Current approach}
  \begin{itemize}
  \item<1-> Formal constraint description.
  \item<2-> Generate some valid IP formulation with additional variables.
  \item<3-> Project onto original variables.
  \item<4-> Classify inequalities.
  \item<5-> Identify coefficient parametrized from initial data.
  \end{itemize}
  This is a multi-year project, with multiple students.
\end{frame}

\begin{frame}
  \frametitle{Example : \texttt{all-different}}
  \begin{displaymath}
    \mathtt{alldifferent}(x_1, \ldots, x_n) \Leftrightarrow x_i \ne x_j
  \end{displaymath}
\end{frame}

\begin{frame}
  \frametitle{Valid IP formulation}
  \begin{align*}
    x_i &\in [0,l]&\text{original}\\
    y_{ij} &\in \{0, 1\}&\text{additional}\\
    x_i &= \sum_j j y_{ij}&\text{assign value}\\
    \sum_i y_{ij} &= 1 & \text{exactly one value}\\
    \sum_j y_{ij} &\le 1&\text{no two on same value}
  \end{align*}
\end{frame}


\begin{frame}
  \frametitle{Example : \texttt{at-least}}
  \begin{displaymath}
    \mathtt{atleast}_m(x_1, \ldots, x_n)=k , \quad x_i\in[0,l]
  \end{displaymath}
  At least $m$ variables out of $n$ take on value $k$.
\end{frame}

\begin{frame}
  \frametitle{Valid IP formulation}
  \begin{align*}
    x_i &\in [0,l]&\text{original}\\
    y_{ij} &\in \{0, 1\}&\text{additional}\\
    x_i &= \sum_j j y_{ij}&\text{assign value}\\
    \sum_i y_{ij} &= 1 & \text{exactly one value}\\
    \sum_j y_{ik} &\ge m&\text{at least $m$ have value $k$}
  \end{align*}
\end{frame}

\begin{frame}
  
  \frametitle{Generate instances}
  \begin{displaymath}
    \mathtt{atleast}_m(x_1, \ldots, x_n)=k 
  \end{displaymath}
  \begin{itemize}
  \item $n$ original variables
  \item $nl$ additional variables
  \item $n+l$ constraints
  \end{itemize}
\end{frame}

\begin{frame}
  \frametitle{Projection}
  \begin{itemize}
  \item $n$ variables
  \item ? constraints
  \end{itemize}
\end{frame}

\begin{frame}
  \frametitle{Classification}
  $atleast_2(x_1,...,x_5) = 3  ; x_i \in [0,..,6]$
\end{frame}

\begin{frame}
  \frametitle{First class of constraints}
  \begin{align*}
    -1.0 x1   -1.0 x2   -1.0 x3   -1.0 x4   -1.0 x5   &\le -6.0 \\
    +1.0 x1   +1.0 x2   +1.0 x3   +1.0 x4   +1.0 x5   &\le 24.0 
  \end{align*}
\end{frame}

\begin{frame}
  \frametitle{Second class}
  \begin{align*}
    -1.0 x1   -1.0 x2   -1.0 x3   -1.0 x4   +1.0 x5   &\le 0.0 \\
    -1.0 x1   -1.0 x2   -1.0 x3   +1.0 x4   -1.0 x5   &\le 0.0 \\
    -1.0 x1   -1.0 x2   +1.0 x3   -1.0 x4   -1.0 x5   &\le 0.0\\
    -1.0 x1   +1.0 x2   -1.0 x3   -1.0 x4   -1.0 x5   &\le 0.0\\
    +1.0 x1   -1.0 x2   -1.0 x3   -1.0 x4   -1.0 x5   &\le 0.0
  \end{align*}
\end{frame}

\begin{frame}
  \frametitle{Third class}
  \begin{align*}
-1.0 x1   +1.0 x2   +1.0 x3   +1.0 x4   +1.0 x5   &\le 18.0 \\
+1.0 x1   -1.0 x2   +1.0 x3   +1.0 x4   +1.0 x5   &\le 18.0 \\
+1.0 x1   +1.0 x2   -1.0 x3   +1.0 x4   +1.0 x5   &\le 18.0 \\
+1.0 x1   +1.0 x2   +1.0 x3   -1.0 x4   +1.0 x5   &\le 18.0 \\
+1.0 x1   +1.0 x2   +1.0 x3   +1.0 x4   -1.0 x5   &\le 18.0 
  \end{align*}
\end{frame}

\begin{frame}
  \frametitle{Fourth class}
  \begin{align*}
-1.0 x1   -1.0 x2   -1.0 x3   +1.0 x4   +1.0 x5   &\le 6.0 \\
-1.0 x1   -1.0 x2   +1.0 x3   -1.0 x4   +1.0 x5   &\le 6.0 \\
-1.0 x1   -1.0 x2   +1.0 x3   +1.0 x4   -1.0 x5   &\le 6.0 \\
-1.0 x1   +1.0 x2   -1.0 x3   -1.0 x4   +1.0 x5   &\le 6.0 \\
-1.0 x1   +1.0 x2   -1.0 x3   +1.0 x4   -1.0 x5   &\le 6.0 \\
-1.0 x1   +1.0 x2   +1.0 x3   -1.0 x4   -1.0 x5   &\le 6.0 \\
+1.0 x1   +1.0 x2   -1.0 x3   -1.0 x4   -1.0 x5   &\le 6.0 \\
+1.0 x1   -1.0 x2   +1.0 x3   -1.0 x4   -1.0 x5   &\le 6.0 \\
+1.0 x1   -1.0 x2   -1.0 x3   +1.0 x4   -1.0 x5   &\le 6.0 \\
+1.0 x1   -1.0 x2   -1.0 x3   -1.0 x4   +1.0 x5   &\le 6.0 
  \end{align*}
\end{frame}

\begin{frame}
  \frametitle{Fifth class}
  \begin{align*}
-1.0 x1   -1.0 x2   +1.0 x3   +1.0 x4   +1.0 x5   &\le 12.0 \\
-1.0 x1   +1.0 x2   -1.0 x3   +1.0 x4   +1.0 x5   &\le 12.0 \\
-1.0 x1   +1.0 x2   +1.0 x3   -1.0 x4   +1.0 x5   &\le 12.0 \\
-1.0 x1   +1.0 x2   +1.0 x3   +1.0 x4   -1.0 x5   &\le 12.0 \\
+1.0 x1   +1.0 x2   +1.0 x3   -1.0 x4   -1.0 x5   &\le 12.0 \\
+1.0 x1   +1.0 x2   -1.0 x3   +1.0 x4   -1.0 x5   &\le 12.0 \\
+1.0 x1   +1.0 x2   -1.0 x3   -1.0 x4   +1.0 x5   &\le 12.0 \\
+1.0 x1   -1.0 x2   +1.0 x3   +1.0 x4   -1.0 x5   &\le 12.0 \\
+1.0 x1   -1.0 x2   +1.0 x3   -1.0 x4   +1.0 x5   &\le 12.0 \\
+1.0 x1   -1.0 x2   -1.0 x3   +1.0 x4   +1.0 x5   &\le 12.0 
  \end{align*}
\end{frame}


\begin{frame}
  \frametitle{Coefficient identification}
  \begin{itemize}
  \item<1-> From
    \begin{align*}
      +1.0 x1   +1.0 x2   -1.0 x3   +1.0 x4   -1.0 x5   &\le 12.0\\
      +1.0 x1   +1.0 x2   +1.0 x3   -2.0 x4   &\le 14.0 \\
      -1.0 x1   +2.0 x2   +2.0 x3   -1.0 x4   +2.0 x5   &\le 28.0 \\
      \ldots
    \end{align*}
  \item<2-> To
    \begin{displaymath}
      k\sum_{i\in J} x_i +(k-l) \sum_{i\in I\setminus J} x_i \le mk^2+(n-m-|J|)lk
    \end{displaymath}
  \end{itemize}
\end{frame}

\begin{frame}
  \frametitle{Current status}
  \begin{itemize}
  \item<1-> IP Formulation is entirely done by hand
  \item<2-> Generation and projection is done via Fourier-Motzkin (Medcoff)
  \item<3-> Classification is done via DBScan (Desjan)
  \item<4-> Identification is done via non-linear least squares (Sibu)
  \end{itemize}
\end{frame}

\begin{frame}
  \frametitle{Properties of library}
  \begin{itemize}
  \item<1-> Written in Lisp
  \item<2-> Symbolic manipulation as much as possible
  \item<3-> Numeric computations over rationals
  \item<4-> On Github soon (TM)
  \item<5-> Includes
    \begin{itemize}
    \item Fourier-Motzkin
    \item Simplex
    \item Least-squares
    \item Classifier
    \end{itemize}
  \end{itemize}
\end{frame}

\begin{frame}
  \frametitle{Focus on projection}
  \begin{itemize}
  \item<1-> We currently use FM with Cernikov's rules.
  \item<2-> Moving towards a parametrized LP.
  \end{itemize}
\end{frame}

\begin{frame}
  \frametitle{How to project efficiently ?}
  Current contenders
  \begin{itemize}
  \item<1-> Fourier-Motzkins with heuristics to discard redundancies (90\%)
  \item<2-> Double description method
  \end{itemize}
\end{frame}

\begin{frame}
  \frametitle{Our situation is particular}
  \begin{itemize}
  \item We project onto a very small subset of variables (from $n^2$ down to $n$).
  \item The eliminated variables are all binaries.
  \end{itemize}
\end{frame}

\begin{frame}
  \frametitle{Small example}
  Consider
  \begin{align*}
    x_1 + x_2 +2x_3 +5 &\ge 0\\
    2x_1 - x_2 - x_3 -4 &\ge 0
  \end{align*}
  Say we want to project onto $(x_1, x_2)$
\end{frame}

\begin{frame}
  \frametitle{Inspired by Farkas lemma}
  Consider $\lambda_i \ge 0$ and
  \begin{displaymath}
    \lambda_1(x_1 + x_2 +2x_3 +5) + \lambda_2 (    2x_1 - x_2 - x_3 -4) \ge 0
  \end{displaymath}
  is a valid inequality.
\end{frame}

\begin{frame}
  \frametitle{Rearrange the coefficients}
  To isolate $x_3$
  \begin{displaymath}
    x_1(\lambda_1 + 2\lambda_2) + x_2(\lambda_1-\lambda_2) + x_3(2\lambda_1-\lambda_2)
  \end{displaymath}
\end{frame}

\begin{frame}
  \frametitle{More generally}
  From a set of constraints
  \begin{displaymath}
    \langle a_i, x \rangle -b_i \ge 0 , \quad i\in I
  \end{displaymath}
  and a set $J$ of variables to eliminate, we consider
  \begin{displaymath}
    \min z(x,\lambda) := \lambda_0 + \sum_i \lambda_i (\langle a_i, x \rangle -b_i)
  \end{displaymath}
  subject to coefficients of variables to eliminate to be zero.
  \begin{center}
    \textbf{A parametrized LP.}
  \end{center}
\end{frame}

\begin{frame}
  \frametitle{Theorem}
  Every solution to the parametrized LP corresponds to a facet of the
  projected polytope.
\end{frame}

\begin{frame}
  \frametitle{Contrast to Fourier-Motzkin}
  \begin{itemize}
  \item<1-> Eliminate all variables at one go.
  \item<2-> No exponential blow-up (except ...)
  \item<3-> Need to solve a parametrized LP.
  \end{itemize}
\end{frame}
\end{document}
